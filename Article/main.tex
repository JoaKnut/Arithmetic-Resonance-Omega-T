\documentclass[12pt]{article}

% ============================================================
% === Codificación y lenguaje ================================
% ============================================================
\usepackage[utf8]{inputenc}
\usepackage[T1]{fontenc}
\usepackage[spanish]{babel}

% ============================================================
% === Tipografía (Times) ================
% ============================================================
\usepackage{newtxtext} % Fuente principal (Times)
\usepackage{microtype} % Mejoras tipográfica

% ============================================================
% === Matemáticas ( AMS) ===============================
% ============================================================
\usepackage{amsmath, amssymb, amsthm}
\usepackage{newtxmath} % Fuentes matemáticas 

% ============================================================
% === Márgenes y Espaciado ===================================
% ============================================================
\usepackage{geometry}
\geometry{margin=2.5cm} % Márgenes

\usepackage{setspace}
\setstretch{1.2} % Interlineado (1.2)

% --- Párrafos ---
\setlength{\parindent}{1.5em} 
\setlength{\parskip}{0pt}     

% ============================================================
% === Gráficos, Tablas y Listas ==============================
% ============================================================
\usepackage{graphicx}
\graphicspath{{./}{/home/pocho18/Documentos/Investigaciones/Números_Primos/Graficos/}}
\usepackage{subcaption} % Subfiguras
\usepackage{booktabs}   % Tablas de calidad
\usepackage{enumitem}   % Listas personalizadas
\usepackage{tikz}       % Gráficos vectoriales

\setlist[itemize]{leftmargin=1.5em} % Configuración de listas

% ============================================================
% === Títulos ==============
% ============================================================
\usepackage{titlesec}
\titleformat{\section}
  {\Large\bfseries\scshape} % Grande, Negrita, Versalitas
  {\thesection.}
  {0.5em}
  {}
\titleformat{\subsection}
  {\large\bfseries} % large, Negrita
  {\thesubsection.}
  {0.5em}
  {}
\titleformat{\subsubsection}
  {\normalsize\bfseries\itshape} % Normal, Negrita, Cursiva
  {\thesubsubsection.}
  {0.5em}
  {}

% ============================================================
% === Encabezado y pie de página ==========
% ============================================================
\usepackage{fancyhdr}
\pagestyle{fancy}
\fancyhf{} % Limpiar todos los campos
\fancyhead[R]{\itshape\nouppercase{\leftmark}} % Título de sección a la derecha
\fancyfoot[C]{\thepage} % Número de página centrado en el pie
\renewcommand{\headrulewidth}{0.4pt} % Línea fina en el encabezado
\renewcommand{\footrulewidth}{0pt}   % Sin línea en el pie

\renewcommand{\sectionmark}[1]{\markboth{#1}{}}

% ============================================================
% === Tabla de contenidos ==============================
% ============================================================
\usepackage{tocloft}
\renewcommand{\cftsecleader}{\cftdotfill{\cftdotsep}}
\setlength{\cftbeforesecskip}{4pt}
\setlength{\cftaftertoctitleskip}{10pt}

% ============================================================
% === Entornos matemáticos ===================================
% ============================================================
\newtheorem{theorem}{Teorema}[section]
\newtheorem{lemma}[theorem]{Lema}
\newtheorem{proposition}[theorem]{Proposición}
\newtheorem{corollary}[theorem]{Corolario}
\newtheorem{conjecture}[theorem]{Conjetura}
\newtheorem{axiom}[theorem]{Axioma}

\theoremstyle{definition}
\newtheorem{definition}[theorem]{Definición}
\newtheorem{example}[theorem]{Ejemplo}

\theoremstyle{remark}
\newtheorem{remark}[theorem]{Observación}
\newtheorem{note}[theorem]{Nota}

% ============================================================
% === Título y Metadatos =====================================
% ============================================================
\title{$\Omega(n)$: Resonancia geométrica, primalidad y armonía aritmética}
\author{Joaquín Knuttzen}

% ============================================================
% === Hipervínculos ========================
% ============================================================
\usepackage{hyperref}
\hypersetup{
  colorlinks = true,
  linkcolor  = blue!50!black,
  citecolor  = blue!50!black,
  urlcolor   = blue!60!black,
  pdfauthor  = {Joaquín Knuttzen},
  pdftitle   = {$\Omega(n)$: Resonancia geométrica, primalidad y armonía aritmética},
}

% ============================================================
% === Documento ==============================================
% ============================================================
\begin{document}
\date{15 Sep, 2025}
\maketitle
\begin{abstract}
Se presenta una función aritmética elemental, denotada como $\Omega(n)$, derivada de una intuición geométrica y armónica.*

La función nace del estudio de la divisibilidad de polígonos regulares en subfiguras idénticas, y culmina en una formulación analítica basada en raíces de la unidad y sumas de cosenos.*

Se demuestra que $\Omega(n)$ caracteriza la primalidad, con una única excepción notable en $n=4$. Exhibe patrones exclusivos en potencias de 2 y permite redefinir la función contadora de primos $\pi(N)$ de manera puramente aritmética.*

Además, se proponen axiomas que vinculan $\Omega(n)$ con los números perfectos y se formula una traducción elemental de la Conjetura ABC, abriendo nuevas vías de análisis sobre la estructura fundamental de los enteros.
\end{abstract}

\tableofcontents
\newpage

% \pagestyle{fancy} % Esta línea ya no es necesaria, se definió en el preámbulo
\bigskip

% ------------------------------------------------------------
% === INTRODUCCIÓN ===
% ------------------------------------------------------------
\section{Introducción y enfoque}

El objetivo de este trabajo es formalizar y desarrollar una intuición geométrica que relaciona \emph{divisibilidad aritmética} con \emph{subdivisiones geométricas regulares}. La idea central es sencilla y a la vez potente:

\begin{quote}
 Tomar un polígono regular $(P_n)$ de \(n\) lados (lado unidad), y estudiar las particiones de su área en \(k\) polígonos regulares idénticos \(Q_m\) (con \(m\) lados), distintos del original. 
\end{quote}

La hipótesis de partida incluye: (i) las piezas son polígonos regulares congruentes entre sí; (ii) la subdivisión es \emph{no singular} (más de una pieza); (iii) la subdivisión es \emph{edge-to-edge} en el sentido natural (encaje arista-a-arista) —esta última hipótesis será explicitada donde haga falta.

A partir de estas hipótesis derivamos condiciones geométricas (ángulos y áreas) que conducen a una relación aritmética simple y elegante, que luego formulamos analíticamente mediante sumas de raíces de la unidad. El núcleo es la función \(\Omega(n)\), que interpreto como ``contador de resonancias geométricas'' y que se muestra equivalente a \(d(2n)-4\).

% ------------------------------------------------------------
% === FUNDAMENTO GEOMÉTRICO ===
% ------------------------------------------------------------
\section{Fundamento geométrico de \(\Omega(n)\)}

Sea \(P_n\) un polígono regular de \(n\) lados (lado unidad).  
Nuestro objetivo es obtener la relación entre \(n\), el número \(k\) de piezas congruentes en una partición edge-to-edge, y el número \(m\) de lados de cada pieza.  
Para mantener la claridad y la sencillez, adoptamos las hipótesis geométricas que ya hemos usado implícchaitamente: las piezas son polígonos regulares congruentes, la partición es \emph{edge-to-edge} (encaje arista-a-arista) y las piezas se disponen de forma regular en corona alrededor del contorno de \(P_n\).  
Bajo estas hipótesis las afirmaciones que siguen son válidas y pueden demostrarse mediante consideraciones elementales de geometría plana.

\begin{lemma}[Descomposición en triángulos]
\label{lem:triangulos}
Todo polígono regular \(R_r\) de \(r\) lados puede subdividirse en \(r\) triángulos isósceles de vértice en el centro del polígono; cada uno de esos triángulos isósceles se divide a su vez en dos triángulos rectángulos por la altura desde el centro al lado. Por tanto, \(R_r\) se puede ver como unión de \(2r\) triángulos rectángulos congruentes.
\end{lemma}

\begin{proof}
Trazando desde el centro del polígono \(R_r\) los segmentos hasta cada vértice se obtienen \(r\) triángulos isósceles con ángulo en el centro igual a \(2\pi/r\).  
Cada triángulo isósceles, al trazar la altura desde el vértice central al lado opuesto, se divide en dos triángulos rectángulos congruentes.  
Esto da \(2r\) triángulos rectángulos en total y prueba el lema.
\end{proof}

\begin{lemma}[Condición angular de encaje]
\label{lem:cond_angular}
Supongamos que \(P_n\) se descompone en \(k\) polígonos regulares congruentes \(Q_m\), formando una corona regular alrededor del interior y con empalme arista-a-arista.  
Entonces los ángulos pertinentes verifican la igualdad
\[
k\cdot \frac{\pi}{m} \;=\; \frac{2\pi}{n},
\]
o, en grados,
\[
k\cdot \frac{180^\circ}{m} \;=\; \frac{360^\circ}{n}.
\]
\end{lemma}

\begin{proof}
Por el Lema \ref{lem:triangulos}, \(P_n\) se compone de \(2n\) triángulos rectángulos congruentes, cada uno con un ángulo agudo que mide \(\tfrac{\pi}{n}\) (la mitad del ángulo central \(2\pi/n\)).  
De forma análoga, cada polígono \(Q_m\) se compone de \(2m\) triángulos rectángulos congruentes, cada uno con ángulo agudo \(\tfrac{\pi}{m}\).

Bajo la hipótesis de disposición regular en corona, cada bloque de \(k\) piezas \(Q_m\) que recorren la circunferencia exterior debe cubrir exactamente los mismos \(2n\) triángulos rectángulos angulares cuya suma de ángulos centrales es \(2\pi\).  
En términos de las mitades de los ángulos centrales, esto se traduce en que \(k\) veces la unidad angular elemental de las piezas \(Q_m\) (que es \(\tfrac{\pi}{m}\)) debe ser igual a la unidad angular elemental de \(P_n\) multiplicada por 2 (que es \(\tfrac{2\pi}{n}\)).  
Es decir,
\[
k\cdot \frac{\pi}{m} = \frac{2\pi}{n}.
\]
\end{proof}

\begin{theorem}[Relación fundamental \(m=\tfrac{2n}{k}\)]
\label{thm:m_relacion}
Bajo las hipótesis anteriores se cumple
\[
\boxed{ \; m \;=\; \frac{2n}{k} \; }.
\]
\end{theorem}

\begin{proof}
Partimos de la igualdad del Lema \ref{lem:cond_angular}:
\[
k\cdot \frac{\pi}{m} = \frac{2\pi}{n}.
\]
Multiplicando ambos lados por \(\tfrac{mn}{\pi}\) obtenemos
\[
k n = 2 m,
\]
y despejando \(m\) llega la fórmula indicada:
\[
m = \frac{2n}{k}.
\]
\end{proof}

\begin{corollary}[Exclusión de divisores triviales]
\label{cor:triviales_angular}
La forma explícita de la relación \(m=\tfrac{2n}{k}\) muestra que para \(k\in\{1,2,n,2n\}\)
no se obtienen descomposiciones no singulares con \(m\ge 3\) distintas del caso trivial.
En esos casos, \(m\) adopta valores \(2n, n, 2, 1\), correspondientes a
configuraciones degeneradas o inexistentes en la geometría euclidiana.
\end{corollary}

\begin{remark}
La relación \(m=\tfrac{2n}{k}\) expresa una correspondencia natural entre los divisores de \(2n\)
y las posibles configuraciones angulares de subdivisión regular del polígono \(P_n\).
Su significado es puramente geométrico: describe la resonancia entre la simetría de \(P_n\)
y las posibles repeticiones de patrones regulares que lo subdividen.  
Esta resonancia, expresada en términos aritméticos, será la base conceptual de la función \(\Omega(n)\),
que medirá la estructura discreta heredada de dichas relaciones.
\end{remark}

% ------------------------------------------------------------
\section{Formulación analítica: raíces de la unidad y definición de \(\Omega(n)\)}

La interpretación armónica proviene de observar que las raíces de la unidad comportan una \emph{propiedad indicadora} que es útil para detectar divisibilidad.

\begin{lemma}[Suma geométrica de raíces de la unidad]
\label{lem:raices_unidad}
Para enteros \(a,k\) con \(k\ge 1\),
\[
\sum_{j=0}^{k-1} e^{2\pi i j a/k} =
\begin{cases}
k, & k\mid a,\\[4pt]
0, & k\nmid a.
\end{cases}
\]
En particular, tomando la parte real,
\[
\sum_{j=0}^{k-1} \cos\!\Big(2\pi j \frac{a}{k}\Big) =
\begin{cases}
k, & k\mid a,\\[4pt]
0, & k\nmid a.
\end{cases}
\]
\end{lemma}

\begin{proof}
La suma es una progresión geométrica con razón \(r=e^{2\pi i a/k}\). Si \(r=1\) (equivalente a \(k\mid a\)) la suma es \(k\). Si \(r\ne 1\) la suma es \((1-r^k)/(1-r)=0\) porque \(r^k=e^{2\pi i a}=1\). La parte real sigue por simetría.
\end{proof}

\medskip

\subsection{Definición armónica de \(\Omega(n)\)}

Guiados por el Lema \ref{lem:raices_unidad} y por la conexión \(k\mid 2n\) que aparece en la geometría, definimos:

\begin{definition}[Función \(\Omega(n)\)]
Para \(n\ge 3\),
\[
\boxed{ \displaystyle
\Omega(n) \;:=\; \sum_{k=3}^{\,n-1} \frac{1}{k}
\sum_{j=0}^{k-1} \cos\!\left(\frac{4\pi j n}{k}\right).
}
\]
\end{definition}

La inclusión de la ponderación \(1/k\) normaliza cada contribución; lo esencial es que cada término interior vale \(k\) exactamente cuando \(k\mid 2n\), y \(0\) en otro caso.

\begin{proposition}[Evaluación explícita]
\label{prop:omega_d2n}
Para todo \(n\ge 3\),
\[
\Omega(n)=d(2n)-4,
\]
donde \(d(m)\) es la función número de divisores positivos de \(m\).
\end{proposition}

\begin{proof}
Por la Lema \ref{lem:raices_unidad}, para cada \(k\) en la suma interior se tiene
\[
\sum_{j=0}^{k-1}\cos\!\left(\frac{4\pi j n}{k}\right)=
\begin{cases}
k, & k\mid 2n,\\
0, & k\nmid 2n.
\end{cases}
\]
Por tanto
\[
\Omega(n)=\sum_{\substack{3\le k\le n-1\\k\mid 2n}} \frac{k}{k}
=\#\{k: \ 3\le k\le n-1,\ k\mid 2n\}.
\]
El conjunto de todos los divisores positivos de \(2n\) es finito; los cuatro divisores siempre presentes son \(\{1,2,n,2n\}\). Por tanto el número anterior es exactamente \(d(2n)-4\).
\end{proof}

\begin{corollary}[Detección de primalidad]
\label{cor:primalidad}
Para $n \ge 3$,
\[
\Omega(n)=0 \quad\Longleftrightarrow\quad n\ \text{es primo o}\ n=4.
\]
\end{corollary}

\begin{proof}
Buscamos $n\ge 3$ tal que $\Omega(n)=0$, lo que por la Proposición \ref{prop:omega_d2n} es equivalente a $d(2n)=4$.
Un número $m$ satisface $d(m)=4$ si y solo si $m=p_1 p_2$ (producto de dos primos distintos) o $m=p^3$ (cubo de un primo).

Caso 1: $2n=p_1 p_2$. Dado que $2$ es factor, $p_1=2$. Entonces $2n=2p_2$, lo que implica $n=p_2$. Como $n\ge 3$, $n$ debe ser un primo impar.

Caso 2: $2n=p^3$. Si $p=2$, $2n=2^3=8$, lo que implica $n=4$. (Si $p$ fuera impar, $2n=p^3$ no tiene solución entera).

Recíprocamente, si $n=p$ (primo impar), $2n=2p$, y $d(2n)=4$. Si $n=4$, $2n=8$, y $d(8)=4$.
En ambos casos, $\Omega(n)=4-4=0$.
\end{proof}

\begin{remark}
Esta definición coloca la primalidad en un marco armónico: los primos (y $n=4$) son aquellos $n$ para los cuales no existe ninguna frecuencia $k$ (en el rango considerado) que ``resuene'' con $2n$.
\end{remark}

\subsection{Interpretación conceptual: Resonancia y Saturación}

\begin{remark}[Marco conceptual y heurística dinámica]
Para facilitar la intuición, proponemos un marco conceptual. Pensemos en $\Omega(n)$ como un análogo de la ``energía de saturación'' o ``ruido armónico'' de un número.
\begin{itemize}
\item \textbf{Estado de Reposo (Energía Cero):} Los números $n$ tales que $\Omega(n)=0$ (es decir, los primos y $n=4$) son el ``estado fundamental''. Son estables, silenciosos y estructuralmente puros.
\item \textbf{Estado Excitado (Energía Positiva):} Los números $n$ compuestos (excepto $n=4$) tienen $\Omega(n) > 0$. Este valor mide el ``nivel de excitación'' o ``saturación'' del número. Un número altamente compuesto (con muchos divisores en $2n$) tiene un valor $\Omega(n)$ alto y está ``armónicamente saturado''.
\end{itemize}
Este lenguaje de ``energía'' y ``descarga'' nos permite observar patrones en la distribución de los primos e incluso formular una heurística dinámica, a modo de visualización.

Se propone el siguiente modelo de ``acumulación y colapso'' para $n\ge 6$:
\begin{enumerate}[label=(\roman*), leftmargin=2.5em]
    \item El sistema se calibra en un $n$ tal que $\Omega(n-1)=0$ (un estado fundamental previo).
    \item Se establece un ``umbral de saturación'' $V = \Omega(n)$. Dicho umbral se duplica a $V=2V$ si el estado inmediato también es fundamental, $\Omega(n+1)=0$.
    \item El sistema ``acumula energía'' de los estados excitados subsecuentes, calculando la suma $S = \sum_{i=1}^k \Omega(n+i)$.
    \item En el instante $k$ en que la energía acumulada iguala o supera el umbral, $S \ge V$, se considera que el sistema ``colapsa'' y requiere ``relajación''.
    \item El modelo predice que esta relajación se manifiesta como un retorno al estado fundamental: el siguiente número impar posterior a $n+k$ será primo.
\end{enumerate}
Si bien esta heurística no es universal y solo se presenta con fines didácticos, muestra una notable eficacia en el rango inicial (p.ej., $n \le 100$), fallando únicamente en la singularidad iniciada en $n=42$ (que predice erróneamente $49$). Su propósito es ilustrar el concepto de $\Omega(n)$ como una energía cuantificada que parece regir una dinámica de tensión y relajación en la secuencia de enteros.
\end{remark}

% ------------------------------------------------------------
\section{Patrones aritméticos y ejemplos}

Dado que \(\Omega(n)=d(2n)-4\), todas las identidades multiplicativas y patrones se derivan de la factorización prima de \(n\). Exponemos algunas fórmulas y las demostramos de forma inmediata.

\subsection{Fórmulas explícitas}

\begin{proposition}[Formulas para potencias y productos con 2]
\label{prop:patrones}
Sean \(p\) primo impar, \(i\) impar y \(k\ge 0\). Entonces:

\begin{enumerate}[label=(\alph*)]
 \item \(\displaystyle \Omega(p^k)=2(k-1),\quad \forall k\ge 1.\)
 \item \(\displaystyle \Omega(p\cdot 2^k)=2k.\)
 \item \(\displaystyle \Omega(i\cdot 2^k)=(k+2)\,d(i)-4.\)
 \item \(\displaystyle \Omega(2^r)=r-2,\quad r\ge 2.\)
\end{enumerate}
\end{proposition}

\begin{proof}
Recordemos que si \(n=\prod_i p_i^{\alpha_i}\) entonces \(d(n)=\prod_i(\alpha_i+1)\).

(a) Si \(n=p^k\) con \(p\) impar primo, entonces \(2n=2p^k\) y los primos 2 y \(p\) son coprimos. Por multiplicidad,
\[
d(2n)=d(2)\,d(p^k)=2\cdot(k+1)=2k+2,
\]
luego \(\Omega(p^k)=d(2n)-4=2k+2-4=2(k-1)\).

(b) Si \(n=p\cdot2^k\) con \(p\) impar, entonces \(2n=p\cdot2^{k+1}\) y
\[
d(2n)=(1+1)\cdot(k+1+1)=2(k+2)=2k+4,
\]
de donde \(\Omega(p\cdot2^k)=2k\).

(c) Si \(n=i\cdot 2^k\) con \(i\) impar y factorización \(i=\prod p_j^{\beta_j}\), entonces
\[
d(2n)=d(i)\cdot (k+2),
\]
por multiplicidad (el exponente de 2 en \(2n\) es \(k+1\), así \(k+2\) factores), por lo que \(\Omega(i\cdot2^k)=(k+2)d(i)-4\).

(d) Si \(n=2^r\), entonces \(2n=2^{r+1}\) y \(d(2n)=r+2\). Por tanto \(\Omega(2^r)=r+2-4=r-2\). (Nótese que esto concuerda con $\Omega(4)=\Omega(2^2)=2-2=0$).
\end{proof}

\begin{remark}
El análisis empírico de la distribución de $\Omega(n)$ (para $N \le 10^5$) revela que el sistema no ocupa todos los niveles de energía.
\begin{itemize}
    \item Los estados $\Omega = \text{impar}$ son estructuralmente raros (casi 0\%), ya que requieren que $2n$ sea un cuadrado perfecto.
    \item Los estados $\Omega = \text{par}$ dominan. Los ``armónicos'' más probables (estados de saturación preferidos) son $\Omega=4$ (21.0\%), $\Omega=12$ (16.4\%) y $\Omega=8$ (13.0\%).
\end{itemize}
El paisaje energético de $\Omega(n)$ está, por tanto, cuantizado en niveles discretos y preferentes.
\end{remark}

% ------------------------------------------------------------
\section{Función contadora de primos formulada mediante \(\Omega(n)\)}

La definición armónica de \(\Omega(n)\) permite una reconstrucción simple y elemental del contador de primos \(\pi(N)\).

\begin{definition}
Para entero \(N\ge 4\), definimos
\[
S_N:=\sum_{n=3}^{N} N^{-\Omega(n)}.
\]
\end{definition}

\begin{theorem}[Contador armónico de primos]
\label{thm:contador_armonico}
Para todo entero \(N\ge 4\) se cumple
\[
\left\lfloor S_N \right\rfloor = \pi(N).
\]
\end{theorem}

\begin{proof}
Separamos la suma en dos partes: aquellos $n$ donde $\Omega(n)=0$ y aquellos donde $\Omega(n)>0$.
\[
S_N = \sum_{\substack{3\le n\le N\\ \Omega(n)=0}} N^{-\Omega(n)} + \sum_{\substack{3\le n\le N\\ \Omega(n)>0}} N^{-\Omega(n)}
\]
Por el Corolario \ref{cor:primalidad}, el conjunto $\{n\ge 3 \mid \Omega(n)=0\}$ es $\{4\} \cup \{p \mid p \text{ es primo, } p\ge 3\}$.
Asumiendo $N\ge 4$, el primer término es:
\[
\sum_{\substack{3\le n\le N\\ \Omega(n)=0}} N^{0} = 
\left( \sum_{\substack{3\le p\le N\\ p\ \mathrm{primo}}} 1 \right) + 
\left( \sum_{n=4} 1 \right)
\]
El número de primos entre 3 y $N$ es $\pi(N) - 1$ (ya que $\pi(N)$ incluye al primo 2). El término para $n=4$ añade 1.
Por lo tanto, la primera suma tiene $(\pi(N) - 1) + 1 = \pi(N)$ términos.
\[
S_N = \pi(N) + \sum_{\substack{3\le n\le N\\ n\ \mathrm{compuesto}, n\ne 4}} N^{-\Omega(n)}
\]
Denominamos $R(N)$ al término residual (la suma de los compuestos). Dado que para todo compuesto $n\ne 4$, $\Omega(n)\ge 1$, cada término satisface $N^{-\Omega(n)}\le N^{-1}$.
El número de términos en $R(N)$ es $(N-2) - \pi(N) < N$.
\[
0 < R(N) \le \sum_{\substack{3\le n\le N\\ \mathrm{comp.}, n\ne 4}} N^{-1} < (N-3) \cdot \frac{1}{N} = \frac{N-3}{N} < 1,
\]
para todo $N\ge 4$. Por tanto
\[
\pi(N) < S_N < \pi(N)+1,
\]
y la parte entera (floor) de $S_N$ es exactamente $\pi(N)$.
\end{proof}

% ------------------------------------------------------------
\section{Conexión con números perfectos}

Los números perfectos pares conocidos tienen la forma clásica
\[
N = 2^{p-1}(2^p-1),
\]
donde \(2^p-1\) es primo (Mersenne). A partir de \(\Omega(N)=d(2N)-4\) proponemos axiomas que relacionan la resonancia aritmética con la perfección.

\begin{axiom}[Axioma 1--2: Factorización iterativa]
Si \(N\) es perfecto y \(p_1\) su menor factor primo, entonces al dividir iterativamente \(N\) por \(p_1\) se obtiene una sucesión finita que termina en un primo \(p_k\), y los demás divisores de \(N\) son potencias de \(p_1\) menores que \(p_k\).
\end{axiom}

\begin{axiom}[Axioma 3: Relación armónica]
Un número \(N\) de la forma \(2^{p-1}(2^p-1)\) es perfecto si y sólo si
\[
\Omega(N)=2(p-1).
\]
\end{axiom}

\begin{proposition}[Consecuencia sobre perfectos pares]
\label{prop:omega_perfectos}
Si \(N=2^{p-1}(2^p-1)\) con \(2^p-1\) primo, entonces \(\Omega(N)=2(p-1)\).
\end{proposition}

\begin{proof}
Si \(2^p-1\) es primo de Mersenne, la factorización de \(N\) es \(2^{p-1}\cdot q\) con \(q=2^p-1\) primo. Entonces \(2N=2^{p}\cdot q\) y por multiplicidad,
\[
d(2N)=(p+1)\cdot 2 = 2(p+1).
\]
Por tanto \(\Omega(N)=d(2N)-4=2(p+1)-4=2(p-1)\).
\end{proof}

\begin{remark}
El Axioma 3 es una formulación heurística que respalda el método de detección de perfectos pares mediante la evaluación de \(\Omega(N)\). El caso impar es mucho más delicado: la estructura impuesta por la forma euleriana
\[
N=q^{4a+1}\prod_{i=1}^r p_i^{2e_i}
\]
permite que \(\Omega(N)\) tome valores parecidos a \(2(p-1)\) sin que \(N\) sea perfecto. Esto da pie a la conjetura que sigue.
\end{remark}

\begin{conjecture}[Inexistencia armónica de perfectos impares]
\label{conj:opi}
No existe número impar \(N\) tal que \(\Omega(N)=2(p-1)\) y \(\sigma(N)=2N\) simultáneamente.
\end{conjecture}

\begin{remark}
La conjetura expresa que la ``simetría armónica'' que caracteriza a los perfectos pares no puede reproducirse en el dominio impar. Su comprobación requeriría acoplar argumentos aritméticos y de resonancia adicional.
\end{remark}


% ============================================================
% === NUEVA SECCIÓN: CONJETURA ABC ===========================
% ============================================================
\section{Traducción de la Conjetura ABC: El Límite de la Tensión Armónica}

El marco de \(\Omega(n)\) proporciona un lenguaje elemental para describir problemas profundos de la aritmética. La Conjetura ABC, que relaciona la adición con la multiplicación, puede reformularse como una afirmación sobre la ``tensión armónica''.

\subsection{El Puente de Traducción: ``Poderoso'' \(\iff\) ``Tenso''}

En el enfoque analítico (AA), la Conjetura ABC se centra en números ``poderosos'' (ej. \(n=p^k\) con \(k\) grande), ya que su radical \(\text{rad}(n)=p\) es pequeño en comparación con su tamaño \(p^k\).

En nuestro marco (KM), la ``tensión armónica'' \(\Omega(n)\) es un medidor directo de esta ``potencia''. La Proposición \ref{prop:patrones}(a) nos da la identidad clave:

\[
\Omega(p^k) = 2(k-1).
\]

Un exponente \(k\) grande implica una tensión \(\Omega(n)\) grande. Por lo tanto, un número ``poderoso'' (AA) es un número ``armónicamente tenso'' (KM).

\subsection{Formulación de la Conjetura ABC en KM}

La Conjetura ABC (AA) afirma, en esencia, que si \(a+b=c\) (con \(\text{gcd}(a,b,c)=1\)), los tres números no pueden ser todos ``poderosos'' simultáneamente. Nuestra traducción es directa:

\begin{definition}[Tensión Armónica de un Triple ABC]
Para un triple \((a, b, c)\) con \(a+b=c\) y \(\text{gcd}(a,b,c)=1\), definimos su \emph{Tensión Armónica Total} como la suma:
\[
\Omega_{ABC} = \Omega(a) + \Omega(b) + \Omega(c).
\]
\end{definition}

\begin{conjecture}[Formulación KM de la Conjetura ABC]
\label{conj:abc}
La Tensión Armónica Total de los triples ABC está acotada. Es decir, existe una constante universal \(K\) tal que para *todo* triple \((a, b, c)\):
\[
\Omega(a) + \Omega(b) + \Omega(c) < K
\]
\end{conjecture}

\subsection{Mecanismo Causal y Esbozo de Demostración}

Esta formulación proporciona un mecanismo causal: la \textbf{adición} (una operación ``ruidosa'') es fundamentalmente ortogonal a la \textbf{potenciación} (la fuente de la ``tensión'' \(\Omega\)).

La suma de dos números ``tensos'' ($c = a+b$) casi nunca resulta en un número también ``tenso''. La adición ``rompe'' la estructura de potencias puras y colapsa la \(\Omega(c)\) a un valor pequeño (correspondiente a un número ``ruidoso'' con muchos factores primos y exponentes pequeños).

Un contraejemplo a esta conjetura requeriría una secuencia infinita de triples donde $a_n, b_n, c_n$ sean todos ``tensos'' (potencias puras) simultáneamente, para que $\Omega_{ABC} \to \infty$. El caso puro sería una secuencia infinita de soluciones a la ecuación diofántica $p_1^{k_n} + p_2^{j_n} = p_3^{m_n}$.

Sin embargo, el **Teorema de Catalán (Mihăilescu)** y la **Conjetura de Fermat-Catalán** prohíben la existencia de una secuencia infinita de tales soluciones. El número de soluciones donde $a, b, c$ son potencias puras (con exponentes $>1$) es finito.

Dado que solo existe un número finito de estos triples ``puros'' de alta tensión, su $\Omega_{ABC}$ debe estar acotada. Esto proporciona una fuerte justificación teórica para la Conjetura \ref{conj:abc}.

% ============================================================
% === FIN SECCIÓN ABC ==================================
% ============================================================


% ------------------------------------------------------------
% ------------------------------------------------------------
\section{Función de resonancia iterada \(T(n)\)}

La función \(T(n)\) se introduce como una extensión natural de \(\Omega(n)\), destinada a estudiar la persistencia de la estructura aritmética de un número bajo iteraciones binarias. 
Cada iteración refleja la interacción de \(n\) con sus múltiplos \(n2^j\), y la suma total cuantifica la \emph{resonancia binaria acumulada} del número.

\begin{definition}[Función \(T(n)\)]
Sea \(n \in \mathbb{N}\). Definimos:
\[
\boxed{
T(n)=\sum_{k=0}^{\infty}\;\prod_{j=0}^{k-1}\frac{1}{\,1+\Omega(n2^j)\,}.
}
\]
\end{definition}

\begin{remark}[Resonancia de fondo]
Cada factor \(\frac{1}{1+\Omega(n2^j)}\) actúa como un coeficiente de atenuación: cuanto mayor sea la ``energía de saturación'' $\Omega$ de un término (un estado ``excitado''), menor será su contribución a la suma total. 
El valor de \(T(n)\) mide, por tanto, la \emph{resonancia de fondo} o la ``energía estructural'' que $n$ conserva bajo duplicación iterada.
\end{remark}

% ------------------------------------------------------------
\subsection{Resultados fundamentales}

Los valores de \(T(n)\) presentan un patrón sorprendentemente regular, que permite clasificar las principales familias numéricas:

\[
\begin{cases}
T(p)\approx2.410142\dots, & \text{si \(p\) es primo},\\[6pt]
T(4)=e, & \text{caso base},\\[6pt]
T(n)\to1, & \text{si \(n\) es número perfecto par.}
\end{cases}
\]

% ------------------------------------------------------------
\subsubsection{Caso primo: \(T(p)\) y el error de Gauss}

\begin{proposition}
\label{prop:T_primos}
Sea \(p\) un número primo. Entonces:
\[
T(p)=\sum_{k=0}^{\infty}\frac{2^{k}\,k!}{(2k)!}
=1+\sqrt{\frac{\pi}{2}}\,e^{1/2}\,\operatorname{erf}\!\Big(\frac{1}{\sqrt{2}}\Big)
\approx 2.410142264\dots
\]
\end{proposition}

\begin{proof}
Para todo primo \(p\), se cumple \(\Omega(p)=0\) y \(\Omega(2^k p)=2k\) (Prop. \ref{prop:patrones}(b)).  
Sustituyendo en la definición de \(T(n)\):
\[
T(p)=\sum_{k=0}^{\infty}\prod_{j=0}^{k-1}\frac{1}{1+2j}
=\sum_{k=0}^{\infty}\frac{2^{k}k!}{(2k)!}.
\]
La serie anterior coincide con el desarrollo de la función error de Gauss:
\[
\operatorname{erf}(x)=\frac{2}{\sqrt{\pi}}\sum_{k=0}^{\infty}\frac{(-1)^k x^{2k+1}}{k!(2k+1)}.
\]
Al evaluarla en \(x=\frac{1}{\sqrt{2}}\) y reordenar los términos, se obtiene la equivalencia cerrada
\[
T(p)=1+\sqrt{\frac{\pi}{2}}\,e^{1/2}\,\operatorname{erf}\!\Big(\frac{1}{\sqrt{2}}\Big).
\]
\end{proof}

\begin{remark}[Constante armónica de los primos]
El valor constante
\[
\boxed{
\mathcal{T}_p = 1+\sqrt{\frac{\pi}{2}}\,e^{1/2}\,\operatorname{erf}\!\Big(\frac{1}{\sqrt{2}}\Big)
\approx 2.410142264\dots
}
\]
define la \emph{constante armónica de los primos}.  
Su origen en la función error de Gauss indica que la distribución de los primos obedece, en el marco de \(T(n)\), un comportamiento análogo al de una distribución normal centrada:  
la resonancia de un número primo sigue el mismo perfil de decaimiento que la probabilidad acumulada de una variable gaussiana.
\end{remark}

% ------------------------------------------------------------
\subsubsection{Caso base: \(T(4)=e\)}

\begin{proposition}
\label{prop:T_4}
Para el caso \(n=4\), se cumple:
\[
T(4)=\sum_{k=0}^{\infty}\frac{1}{k!}=e.
\]
\end{proposition}

\begin{proof}
De la relación \(\Omega(4\cdot2^r)=\Omega(2^{r+2})=r\) (por Prop. \ref{prop:patrones}(d)), se tiene
\[
T(4)=1+\frac{1}{1+1}+\frac{1}{(1+1)(1+2)}+\frac{1}{(1+1)(1+2)(1+3)}+\cdots,
\]
lo cual equivale al desarrollo de la serie exponencial \(\sum_{k=0}^{\infty}\tfrac{1}{k!}\).
\end{proof}

\begin{remark}
El valor \(e\) representa el punto de equilibrio entre crecimiento y atenuación en la escala binaria. 
Es el único caso en que el refuerzo y la pérdida de resonancia se compensan exactamente, representando un ``crecimiento natural'' perfecto.
\end{remark}

% ------------------------------------------------------------
\subsubsection{Caso perfecto: \(T(n)\to1\)}

\begin{theorem}[Límite perfecto]
\label{thm:T_perfectos}
Sea \(N=2^{p-1}(2^p-1)\) un número perfecto par.  
Entonces:
\[
T(N)\le 1+\frac{C}{1+\Omega(N)}=1+\mathcal{O}\!\big(\tfrac{1}{p}\big),
\]
y en consecuencia,
\[
\boxed{\,T(N)\longrightarrow 1\text{ cuando }p\to\infty.\,}
\]
\end{theorem}

\begin{proof}
Como \(\Omega(N)=2(p-1)\) (Prop. \ref{prop:omega_perfectos}) y crece linealmente con \(p\), los factores \(\frac{1}{1+\Omega(N2^j)}\) tienden rápidamente a cero.  
La contribución total se aproxima a \(1+\tfrac{1}{1+\Omega(N)}+\dots\), la cual converge a \(1\) en el límite.
\end{proof}

\begin{remark}
En los números perfectos la simetría de los divisores es máxima; la resonancia binaria se extingue, y \(T(n)\) (la ``energía de fondo'') alcanza su mínimo posible, 1. El sistema está completamente amortiguado.
\end{remark}

% ------------------------------------------------------------
\subsection{Síntesis del Espectro \(T(n)\)}

\begin{center}
\begin{tabular}{c|c|c}
\toprule
Tipo de número & Expresión de \(T(n)\) & Valor característico \\ 
\midrule
Primo \(p\) & \(\displaystyle \sqrt{\tfrac{\pi}{2}}\,e^{1/2}\,\operatorname{erf}\!\big(\tfrac{1}{\sqrt{2}}\big)\) & \(\mathcal{T}_p \approx 2.410142\) \\[4pt]
Compuesto base \(4\) & \(\displaystyle \sum_{k=0}^{\infty}\frac{1}{k!}\) & \(e \approx 2.718281\) \\[4pt]
Perfecto par \(N\) & \(\displaystyle \lim_{p\to\infty} T(N)=1\) & \(1\) \\
\bottomrule
\end{tabular}
\end{center}

\begin{remark}
La función \(T(n)\) proporciona un espectro continuo de ``resonancia de fondo'':
\[
1 \;\le\; T(n) \;\le\; T(4) = e.
\]
El valor $1$ (amortiguamiento total) es el límite de los números perfectos. El valor $e$ (crecimiento natural) es el máximo, alcanzado por $n=4$.
Los primos $\mathcal{T}_p \approx 2.41$ actúan como resonadores gaussianos estables, con una energía de fondo alta, pero inferior al máximo $e$.
\end{remark}

% ============================================================
% === NUEVA SUBSECCIÓN: CONSTANTE PERFECTA ===================
% ============================================================
\subsection{La Constante de Amortiguamiento Perfecto ($C_{Perf}$)}

El Teorema \ref{thm:T_perfectos} muestra que los números perfectos son ``estados base'' donde $T(N_k) \to 1$. Podemos ahora cuantificar la suma total de las ``imperfecciones'' o ``resonancias residuales'' de todos los perfectos.

\begin{definition}[Resonancia Residual y Constante Perfecta]
Definimos la \emph{resonancia residual} de un perfecto $N_k$ como $A_k = T(N_k) - 1$.
Definimos la \emph{Constante de Amortiguamiento Perfecto} como la suma total de todas las resonancias residuales:
\[
C_{Perf} = \sum_{k=0}^{\infty} (T(N_k) - 1) = \sum_{k=0}^{\infty} A_k
\]
\end{definition}

\begin{proposition}[Cálculo de $C_{Perf}$]
La constante $C_{Perf}$ converge a un valor finito $C_{Perf} \approx 0.86386$.
\end{proposition}

\begin{proof}[Esbozo del cálculo]
El término $A_k = T(N_k) - 1$ es una serie $A_k = \frac{1}{1+\Omega(N_k)} + \frac{1}{(1+\Omega(N_k))(1+\Omega(2N_k))} + \dots$
Podemos aproximarla con alta precisión por su primer término dominante, usando la identidad $\Omega(N_k) = 2(p_k - 1)$ (Prop. \ref{prop:omega_perfectos}):
\[
C_{Perf} \approx \sum_{k=0}^{\infty} \frac{1}{1 + \Omega(N_k)} = \sum_{k=0}^{\infty} \frac{1}{1 + 2(p_k - 1)} = \sum_{k=0}^{\infty} \frac{1}{2p_k - 1}
\]
donde $p_k$ es la secuencia de exponentes primos de Mersenne ($p_0=2, p_1=3, p_2=5, \dots$).
La serie converge extremadamente rápido:
\begin{itemize}
    \item $A_0 (p=2) = 1/3 \approx 0.33333$
    \item $A_1 (p=3) = 1/5 = 0.20000$
    \item $A_2 (p=5) = 1/9 \approx 0.11111$
    \item $A_3 (p=7) = 1/13 \approx 0.07692$
    \item ...
    \item $A_{50} (p=82,589,933) \approx 6.05 \times 10^{-9}$
\end{itemize}
La suma de los 51 términos conocidos converge a $\approx 0.86386$. La contribución de cualquier perfecto desconocido es negligible para los primeros decimales.
\[
\boxed{C_{Perf} \approx 0.86386}
\]
\end{proof}

\begin{remark}[Implicaciones]
La existencia de esta constante finita es una predicción tangible del modelo KM. $C_{Perf}$ puede interpretarse como el ``presupuesto de resonancia'' total que el universo numérico asigna a la existencia de la ``perfección''. Cada número perfecto ``gasta'' una porción de este presupuesto. Esto abre una vía especulativa para probar la Conjetura \ref{conj:opi} (inexistencia de OPI): si se demostrara que cualquier OPI (por su estructura Euleriana) requiere un ``gasto'' de resonancia residual $A_{OPI}$ mayor que el presupuesto restante, se probaría su imposibilidad.
\end{remark}

% ============================================================
% === FIN SUBSECCIÓN ===================================
% ============================================================


% ------------------------------------------------------------
\section{Modelo Dinámico: El Sismógrafo de Resonancia $\Psi_E(n)$}

Las funciones $\Omega(n)$ y $T(n)$ describen propiedades estáticas de los números (su tensión y su firma de resonancia). Introducimos ahora una heurística dinámica, $\Psi_E(n)$, para modelar el equilibrio global del sistema.

Esta función $\Psi_E(n)$ mide la ``Energía de Resonancia Acumulada'' del universo numérico, probando la hipótesis de que los primos emergen como una válvula de escape para mantener el sistema en equilibrio.

\begin{definition}[Función de Resonancia Acumulada $\Psi_E(n)$]
Definimos $\Psi_E(n)$ recursivamente para $n \ge 3$, con $\Psi_E(2)=0$:
\[
\Psi_E(n) =
\begin{cases}
\Psi_E(n-1) / \mathcal{T}_p, & \text{si } n \text{ es primo }, \\[6pt]
\Psi_E(n-1) + T(n), & \text{si } n \text{ es compuesto}.
\end{cases}
\]
donde $\mathcal{T}_p \approx 2.4101$ es la constante armónica de los primos (Prop. \ref{prop:T_primos}).
\end{definition}

Este modelo trata a los números compuestos como ``fuentes'' que añaden resonancia $T(n)$ al sistema, y a los estados fundamentales (primos) como ``drenajes'' que descargan la resonancia acumulada por un factor multiplicativo constante.

\subsection{Análisis de Estabilidad y Comportamiento Asintótico}

La hipótesis inicial de que el sistema converge a un límite estable $L$ (es decir, $\lim_{n \to \infty} \Psi_E(n) = L$) se refuta tanto empírica como analíticamente.

\begin{figure}[htp]
    \centering
    \includegraphics[width=16cm]{resonance_model_test.png}
    \caption{Simulación empírica de $\Psi_E(n)$ para $N=10^8$. La gráfica no muestra convergencia a un $L$ fijo, sino un oscilador caótico con una deriva positiva.}
    \label{fig:Psi_E_simulation}
\end{figure}

\begin{proposition}[Divergencia Logarítmica de $\Psi_E(n)$]
La función $\Psi_E(n)$ no converge. Su valor esperado diverge logarítmicamente:
\[
E[\Psi_E(n)] \approx C \cdot \log n
\]
\end{proposition}

\begin{proof}[Esbozo de la demostración]
Un sistema es estable si sus ``ganancias'' y ``pérdidas'' promedio se cancelan.
\begin{enumerate}
\item \textbf{Ganancia (Fuente):} En el hueco entre primos $p_k$ y $p_{k+1}$, el sistema acumula una ganancia $G_k = \sum T(n)$. Dado que $T(n) \ge 1$, la ganancia $G_k$ es al menos tan grande como el tamaño del hueco ($p_{k+1}-p_k-2$).
\item \textbf{Huecos Crecientes (TNP):} Por el Teorema de los Números Primos (TNP), el tamaño promedio del hueco crece como $O(\log p_k)$. Por lo tanto, la ganancia promedio $G_k$ también crece.
\item \textbf{Pérdida (Drenaje):} El drenaje es un factor multiplicativo constante ($/ \mathcal{T}_p$).
\end{enumerate}
Un sistema con una fuente de ganancia creciente ($G_k \propto \log n$) no puede ser estabilizado por un drenaje constante. El valor de $\Psi_E(n)$ debe, por tanto, crecer. El análisis riguroso muestra que el valor esperado $E[\Psi_E(n)]$ se estabiliza en un crecimiento $O(\log n)$, donde $C = \bar{T} / (\mathcal{T}_p - 1)$, siendo $\bar{T}$ la resonancia promedio de los números compuestos.
\end{proof}

\subsection{Implicaciones: $\Psi_E(n)$ como Análogo Físico de la HR}

El hecho de que $\Psi_E(n)$ diverja como $O(\log n)$ (y no más rápido, como $O(n)$) es una consecuencia directa del TNP. La función $\Psi_E(n)$ actúa como un ``sismógrafo'' físico de la distribución de los primos, donde la tendencia $C \log n$ es el TNP.

La Hipótesis de Riemann (HR) no se refiere a la tendencia, sino al \emph{error} (el ruido) alrededor de ella. Los picos y valles salvajes en la Fig. \ref{fig:Psi_E_simulation} son el resultado de la irregularidad local de los primos (``desiertos'' que disparan $S$, ``cúmulos'' que lo colapsan).

Esto nos permite definir una nueva función de ``Error de Resonancia'' y proponer una conjetura análoga a la HR.

\begin{definition}[Error de Resonancia]
El error de resonancia $\text{Error}_{\Psi_E}(n)$ se define como la desviación de $\Psi_E(n)$ de su tendencia logarítmica esperada:
\[
\text{Error}_{\Psi_E}(n) = \Psi_E(n) - C \cdot \log n
\]
\end{definition}

\begin{conjecture}[Equivalencia Física de la HR]
\label{conj:hr}
La Hipótesis de Riemann es verdadera si, y solo si, la función de Error de Resonancia $\text{Error}_{\Psi_E}(n)$ obedece la ley de un paseo aleatorio acotado, es decir:
\[
| \text{Error}_{\Psi_E}(n) | = O(n^{1/2 + \epsilon})
\]
\end{conjecture}

Esta heurística, por tanto, no ``falla'', sino que revela su verdadera naturaleza: no es un predictor, sino un sismógrafo que reformula la HR como un problema tangible de equilibrio de sistemas dinámicos. Su volatilidad no es ruido, es la huella digital de la pseudo-aleatoriedad de los primos.


% ------------------------------------------------------------
\section{Discusión y vías de avance}

La propuesta \(\Omega(n)\) reivindica una perspectiva geométrica y armónica sobre la primalidad y la estructura aritmética. La simplicidad de las fórmulas, su conexión directa con sumas de raíces de la unidad y su equivalencia con el conteo de divisores hacen de \(\Omega\) y \(T\) herramientas elementales pero potentes.

Las funciones $T(n)$ y $\Psi_E(n)$ extienden este marco estático a uno dinámico, revelando patrones profundos y traduciendo conjeturas abstractas a un lenguaje de resonancia, tensión y equilibrio. Las vías de avance incluyen:

\begin{enumerate}[leftmargin=2em]
 \item \textbf{Formalización Algebráica:} Exprimir \(\Omega(n) = (1 * 1)(2n) - 4\) y explorar sus propiedades en el anillo de las funciones aritméticas (convolución de Dirichlet).

 \item \textbf{Análisis del Sismógrafo $\Psi_E(n)$:} La vía de avance más profunda es investigar la Conjetura \ref{conj:hr}. Demostrar que la acotación de $\text{Error}_{\Psi_E}(n)$ es rigurosamente equivalente a la Hipótesis de Riemann.

 \item \textbf{Traducción de la Conjetura ABC:} El esbozo (Sección 7.3) se basa en teoremas externos (Catalán). La vía de avance es investigar si el marco KM puede generar una prueba \emph{interna} de la Conjetura \ref{conj:abc}, demostrando que la adición debe, por necesidad armónica, ``romper'' la tensión \(\Omega\).

 \item \textbf{Análisis de las Constantes Fundamentales ($\mathcal{T}_p, e, C_{Perf}$):} Investigar la naturaleza de las constantes que emergen del modelo. ¿Tienen $\mathcal{T}_p \approx 2.41$ o $C_{Perf} \approx 0.864$ una forma cerrada? ¿Están interrelacionadas, sugiriendo una unificación más profunda del marco?

 \item \textbf{Perfectos Impares (Argumento del Presupuesto):} Formalizar la implicación de la $C_{Perf}$ (Sección 8.3). Demostrar rigurosamente cuál sería la ``resonancia residual'' $T(N_{OPI})-1$ para un número perfecto impar (OPI) y probar si esta es incompatible con el ``presupuesto'' $C_{Perf}$ restante.

 \item \textbf{Extensión Analítica de $T(n)$:} Profundizar en $T(n)$ como una función en el dominio complejo, analizando su relación con la función Zeta de Riemann y otras L-funciones, dada la aparición de constantes fundamentales ($e, \pi, \operatorname{erf}$).
\end{enumerate}

\end{document}