\documentclass[12pt]{article}

% ------------------------------------------------------------
% Encoding and Language
% ------------------------------------------------------------
\usepackage[T1]{fontenc}
\usepackage[utf8]{inputenc} % remove if using LuaLaTeX or XeLaTeX
\usepackage[english]{babel} % Changed from spanish to english

% ------------------------------------------------------------
% Margins and Spacing
% ------------------------------------------------------------
\usepackage{geometry}
\geometry{margin=2.5cm}

\usepackage{setspace}
\setstretch{1.2}

\setlength{\parindent}{1.5em}
\setlength{\parskip}{4pt}

% ------------------------------------------------------------
% Graphics, Tables, and Lists
% ------------------------------------------------------------
\usepackage{graphicx}
\usepackage{subcaption}
\usepackage{booktabs}
\usepackage{enumitem}

\setlist[itemize]{leftmargin=*, itemsep=2pt}
\setlist[enumerate]{leftmargin=*, itemsep=2pt}

\usepackage{tikz}

% ------------------------------------------------------------
% Titles and Headers
% ------------------------------------------------------------
\usepackage{titlesec}

\titleformat{\section}{\Large\bfseries\scshape}{\thesection.}{0.6em}{}
\titleformat{\subsection}{\large\bfseries}{\thesubsection.}{0.5em}{}

\usepackage{fancyhdr}
\pagestyle{fancy}
\fancyhf{}
\fancyhead[R]{\itshape\leftmark}
\fancyfoot[C]{\thepage}
\renewcommand{\headrulewidth}{0.4pt}

% ------------------------------------------------------------
% Table of Contents
% ------------------------------------------------------------
\usepackage{tocloft}
\renewcommand{\cftsecleader}{\cftdotfill{\cftdotsep}}
\setlength{\cftbeforesecskip}{4pt}
\setlength{\cftaftertoctitleskip}{10pt}

% ------------------------------------------------------------
% Mathematics and Theorems
% ------------------------------------------------------------
\usepackage{amsmath, amssymb, amsthm}
\usepackage{newtxtext}
\usepackage{newtxmath}
\usepackage{microtype}

\theoremstyle{plain} % Italic text (For strong statements)
\newtheorem{theorem}{Theorem}[section]
\newtheorem{lemma}[theorem]{Lemma}
\newtheorem{proposition}[theorem]{Proposition}
\newtheorem{corollary}[theorem]{Corollary}
\newtheorem{conjecture}[theorem]{Conjecture} 

\theoremstyle{definition} % Normal text (For definitions and axioms)
\newtheorem{definition}[theorem]{Definition}
\newtheorem{axiom}[theorem]{Axiom}
\newtheorem{example}[theorem]{Example}

\theoremstyle{remark} % Normal text (For notes and remarks)
\newtheorem{remark}[theorem]{Remark}
\newtheorem{note}[theorem]{Note}
% ------------------------------------------------------------
% Hyperlinks
% ------------------------------------------------------------
\usepackage{hyperref}
\hypersetup{
    colorlinks = true,
    linkcolor  = blue!50!black,
    citecolor  = blue!50!black,
    urlcolor   = blue!60!black,
    pdfauthor  = {Joaquín Knuttzen},
    pdftitle   = {Geometric Resonance in Integers},
}

% ============================================================
% === DOCUMENT ===============================================
% ============================================================
\begin{document}

\title{\textbf{Geometric Resonance in Integers}\\ \large A harmonic derivation of the divisor function and its spectral dynamics}
\author{Joaquín Knuttzen}
\date{09/20/2025}

\maketitle

\begin{abstract}
This paper explores an isomorphism between the geometry of regular polygon subdivision and arithmetic divisor theory. The function $\Omega(n)$ is defined based on sums of roots of unity, formally demonstrating its equivalence with $d(2n)-4$. From this rigorous basis, advanced analytic structures are derived: an iterated resonance function $T(n)$ that connects prime numbers with the Gaussian error function, and a deterministic "frequency seed" $\Lambda_{MF}$ based on parity. The first part of this article establishes the foundations and proven identities, while the second part explores heuristic dynamic models applied to the distribution of prime numbers.
\end{abstract}

\tableofcontents
\newpage

% ============================================================
% === PART I: ANALYTICAL FOUNDATIONS =========================
% ============================================================
\part{Analytical and Geometric Foundations}
\begin{center}
    \small\textit{In this section, rigorous definitions are established and the fundamental arithmetic identities of the model are demonstrated, providing a solid basis for the subsequent discussion.}
\end{center}

% ------------------------------------------------------------
% SECTION 1
% ------------------------------------------------------------
\section{Introduction and Geometric Foundation}

The starting point of this research is a classic tessellation problem: the study of regular partitions of a polygon. Let $P_n$ be a regular polygon with $n$ sides. We ask under what arithmetic conditions $P_n$ can be subdivided into $k$ identical regular polygons $Q_m$ with $m$ sides, arranged in a regular angular configuration (edge-to-edge).

To formalize this intuition, we resort to elementary triangular decomposition.

\begin{lemma}[Triangular decomposition and angular equivalence]
\label{lem:triangulos}
Every regular polygon with $r$ sides can be decomposed into $2r$ congruent right triangles, whose acute vertices coincide at the center of the polygon. The measure of the central angle of each elementary triangle is $\pi/r$.
\end{lemma}

\begin{proof}
We divide the polygon into $r$ isosceles triangles by connecting the center with the vertices. Each isosceles triangle has a central angle of $2\pi/r$. By drawing the altitude (apothem), each isosceles triangle is divided into two identical right triangles, bisecting the central angle. Therefore, we obtain $2r$ triangles with angle $\pi/r$.
\end{proof}

This decomposition allows us to establish a necessary condition for the assembly of the pieces $Q_m$ inside $P_n$.

\begin{theorem}[Fundamental Geometric Relation]
\label{thm:relacion_geometrica}
Suppose a regular corona subdivision where $k$ polygons $Q_m$ cover the inner perimeter of $P_n$. The necessary angular fitting condition implies the relation:
\[
\boxed{ m = \frac{2n}{k} }
\]
\end{theorem}

\begin{proof}
Let us consider the sum of the central angles. To cover the full angle ($2\pi$) of the larger polygon $P_n$ using $k$ copies of $Q_m$, the sum of the internal semi-angles contributed by the pieces must be congruent with the geometry of the container.
\begin{enumerate}
    \item The elementary angle of the larger polygon $P_n$ is $\pi/n$. The total angle is $2\pi$, which equals $2n$ elementary units.
    \item The elementary angle of the piece $Q_m$ is $\pi/m$.
    \item If we arrange $k$ pieces, the angular sum must satisfy the proportion:
    \[
    k \cdot \frac{\pi}{m} = \frac{2\pi}{n}
    \]
\end{enumerate}
Multiplying both sides by $\frac{mn}{\pi}$, we obtain $kn = 2m$. Solving for $m$, we arrive at $m = \frac{2n}{k}$.
\end{proof}

\begin{remark}[Physical Interpretation]
The equation $m=2n/k$ suggests that the geometry of the subdivision is not continuous, but quantized. Solutions only exist when $k$ is a divisor of $2n$. Geometrically, $k=1, k=2$, $k=n$, or $k=2n$ usually produce degenerate or trivial solutions, which motivates the need to filter these cases in the subsequent analytical definition.
\end{remark}

% ------------------------------------------------------------
% SECTION 2
% ------------------------------------------------------------
\section{The Function $\Omega(n)$: Analytical Definition and Equivalence}

To translate the divisibility condition $k \mid 2n$ into the language of complex analysis, we use the orthogonality property of the roots of unity. This allows us to construct a "probe" function that detects resonances.

\begin{definition}[Function $\Omega(n)$]
For every integer $n \ge 3$, we define the geometric resonance function as the weighted sum:
\[
\Omega(n) := \sum_{k=3}^{n-1} \frac{1}{k} \sum_{j=0}^{k-1} \cos\left(\frac{4\pi j n}{k}\right).
\]
\end{definition}

The inner term is a sum of cosines that acts as an indicator function. We proceed to demonstrate exactly what this function counts.

\begin{theorem}[Equivalence Theorem]
The function $\Omega(n)$ is analytically equivalent to the shifted divisor function:
\[
\boxed{ \Omega(n) = d(2n) - 4 }
\]
where $d(n)$ represents the number of positive divisors of $n$.
\end{theorem}

\begin{proof}
Let us analyze the inner sum $S_k = \sum_{j=0}^{k-1} \cos(4\pi j n / k)$. This is the real part of a geometric series of roots of unity $e^{i \theta}$ with $\theta = 4\pi n / k$.
\begin{itemize}
    \item \textbf{Case 1: Resonance ($k \mid 2n$).} If $k$ divides $2n$, then $\frac{2n}{k}$ is an integer, and the argument $2\pi j (\frac{2n}{k})$ is a multiple of $2\pi$. Therefore, $\cos(\cdot) = 1$ for all $j$. The sum is $\sum_{j=0}^{k-1} 1 = k$.
    \item \textbf{Case 2: Dissonance ($k \nmid 2n$).} If $k$ does not divide $2n$, the vectors in the complex plane cancel out by rotational symmetry. The sum is 0.
\end{itemize}
Substituting this into the definition of $\Omega(n)$:
\[
\Omega(n) = \sum_{k=3}^{n-1} \frac{1}{k} \cdot (k \cdot \mathbf{1}_{k \mid 2n}) = \sum_{k=3}^{n-1} \mathbf{1}_{k \mid 2n}.
\]
The function simply counts how many integers $k$ in the range $[3, n-1]$ are divisors of $2n$.
The total set of divisors of $2n$ always includes at least four elements that fall outside this range or are trivial to the geometry:
\begin{enumerate}
    \item $1$ and $2$ (always less than 3).
    \item $n$ and $2n$ (always greater than $n-1$, given $n \ge 3$).
\end{enumerate}
Therefore, $\Omega(n)$ counts all divisors of $2n$ except these four. We conclude that $\Omega(n) = d(2n) - 4$.
\end{proof}

\begin{corollary}[Primality Filter]
\label{cor:primalidad}
For $n > 4$, the following holds:
\[
\Omega(n) = 0 \iff n \text{ is a prime number}.
\]
\end{corollary}

\begin{proof}
If $\Omega(n)=0$, then $d(2n)=4$. We know that $d(2n)=4$ occurs if and only if $2n$ is the product of two distinct primes ($2 \cdot p$) or a cube of a prime ($p^3$).
Since $2n$ contains the factor 2, the options are $2n = 2 \cdot p$ (with $p$ being an odd prime) or $2n = 2^3 = 8$.
\begin{itemize}
    \item If $2n = 2p$, then $n=p$ is prime.
    \item If $2n = 8$, then $n=4$. (This is the case $\Omega(4)=d(8)-4=0$, the only even exception).
\end{itemize}
Thus, for $n > 4$, the absence of internal resonance ($\Omega=0$) implies strict primality.
\end{proof}

% ------------------------------------------------------------
% SECTION 3
% ------------------------------------------------------------
\section{Arithmetic Patterns and Resonance Structure}

The fundamental identity $\Omega(n) = d(2n)-4$ shifts the geometric problem to the domain of multiplicative number theory. This allows us to catalogue the exact behavior of the resonance $\Omega(n)$ based on the prime factorization of $n$.

Below, we present the identities governing the behavior of $\Omega$ across different classes of integers.

\begin{proposition}[Spectral Identities of $\Omega(n)$]
\label{prop:patrones}
Let $k \ge 1$ be an integer, $p$ an odd prime number, and $i$ any odd integer ($i \ge 1$). The following identities hold:

\begin{enumerate}[label=(\alph*)]
    \item \textbf{Evolution in Powers of 2:}
    \[ \Omega(2^k) = k - 2 \]
    
    \item \textbf{Resonance in Powers of Odd Primes:}
    \[ \Omega(p^k) = 2(k - 1) \]
    
    \item \textbf{Prime-Binary Interaction (Simple Mixture):}
    \[ \Omega(p \cdot 2^k) = 2k \]
    
    \item \textbf{Odd-Binary Interaction (General Mixture):}
    \[ \Omega(i \cdot 2^k) = (k + 2)d(i) - 4 \]
\end{enumerate}
\end{proposition}

\begin{proof}
The proofs follow directly from the multiplicative property of the divisor function $d(n)$, applied to the doubled argument $2n$.

\textbf{(a) Case $n = 2^k$:}
Double the number is $2n = 2^{k+1}$. The divisor function for a prime power is simply the exponent plus one:
\[ d(2n) = d(2^{k+1}) = (k+1) + 1 = k+2. \]
Substituting into the definition of equivalence:
\[ \Omega(2^k) = (k+2) - 4 = k - 2. \]

\textbf{(b) Case $n = p^k$ ($p$ odd):}
Double the number is $2n = 2^1 \cdot p^k$. Since $\gcd(2, p)=1$, the divisor function separates:
\[ d(2n) = d(2^1) \cdot d(p^k) = (1+1)(k+1) = 2k + 2. \]
Substituting:
\[ \Omega(p^k) = (2k+2) - 4 = 2k - 2 = 2(k-1). \]

\textbf{(c) Case $n = p \cdot 2^k$ ($p$ odd):}
Double is $2n = p^1 \cdot 2^{k+1}$. By multiplicity:
\[ d(2n) = d(p^1) \cdot d(2^{k+1}) = 2 \cdot (k+2) = 2k + 4. \]
Substituting:
\[ \Omega(p \cdot 2^k) = (2k+4) - 4 = 2k. \]

\textbf{(d) General Case $n = i \cdot 2^k$ ($i$ odd):}
Double is $2n = i \cdot 2^{k+1}$. Given that $i$ is odd, it is coprime to 2, so the divisor function is multiplicative:
\[ d(2n) = d(i) \cdot d(2^{k+1}) = d(i) \cdot (k+2). \]
Finally, we subtract the 4 trivial geometric divisors:
\[ \Omega(i \cdot 2^k) = (k+2)d(i) - 4. \]
\end{proof}

\begin{remark}[Observation on Structure]
The general identity (d) is particularly revealing. It shows that the resonance $\Omega$ of an even number ($i \cdot 2^k$) is a linear amplification of the quantity of divisors of its odd part ($d(i)$), scaled by the binary exponent $(k+2)$.
Note that identity (c) is a special case of (d) where $i=p$ (and thus $d(i)=2$):
\[ (k+2)(2) - 4 = 2k + 4 - 4 = 2k. \]
\end{remark}

% ------------------------------------------------------------
% SECTION 4
% ------------------------------------------------------------
\section{Harmonic Formulation of the Function $\pi(N)$}

Given that $\Omega(n)$ nullifies primes (and 4), we can construct a prime counter $\pi(N)$ that does not depend on explicit sieves, but on the sum of resonant "silences".

\begin{theorem}[Exact Harmonic Counter]
For every integer $N \ge 4$, the number of primes less than or equal to $N$ is given by:
\[
\pi(N) = \left\lfloor \sum_{n=3}^{N} N^{-\Omega(n)} \right\rfloor
\]
\end{theorem}

\begin{proof}
Let us analyze the behavior of each term $t_n = N^{-\Omega(n)}$ in the sum $S_N$.
We divide the summation range $[3, N]$ into two disjoint sets:
\begin{itemize}
    \item \textbf{Primal Set (and n=4):} $P = \{n : \Omega(n)=0\}$. For these $n$, $t_n = N^0 = 1$.
    This set contains the number 4 and all odd primes up to $N$. The number of elements is $1 + (\pi(N)-1) = \pi(N)$ (the -1 discounts the prime 2, which is not in the sum, but the +1 adds 4, compensating exactly).
    \item \textbf{Composite Set:} $C = \{n : \Omega(n) \ge 1\}$. For these $n$, the term is $t_n \le N^{-1}$.
    The sum of these residues is strictly less than 1:
    \[
    \sum_{n \in C} N^{-\Omega(n)} \le \sum_{n \in C} \frac{1}{N} = \frac{|C|}{N} < \frac{N-3}{N} < 1.
    \]
\end{itemize}
Therefore, the total sum is $S_N = \pi(N) + \epsilon$, with $0 \le \epsilon < 1$. By applying the floor function $\lfloor \cdot \rfloor$, we eliminate the decimal residue coming from the composites, obtaining exactly $\pi(N)$.
\end{proof}

% ------------------------------------------------------------
% SECTION 5
% ------------------------------------------------------------
\section{The Iterated Resonance Function $T(n)$}

The linear property of $\Omega$ under doubling (Prop. \ref{prop:patrones}) suggests defining a function that measures the structural "stability" of a number against iterated multiplication by powers of 2. This function quantifies the number's resistance to generating new divisors under the doubling operation.

\begin{definition}[Function $T(n)$]
We define Iterated Resonance as the infinite series of damped products:
\[
T(n) := \sum_{k=0}^{\infty} \prod_{j=0}^{k-1} \frac{1}{1+\Omega(n \cdot 2^j)}.
\]
\end{definition}

This definition, seemingly complex, collapses into universal constants for fundamental families of integers.

\begin{theorem}[Gaussian Connection of Primes]
For every prime number $p$, the value of $T(p)$ is a universal constant related to the Gaussian error function:
\[
T(p) = 1 + \sqrt{\frac{\pi}{2}} e^{1/2} \operatorname{erf}\left(\frac{1}{\sqrt{2}}\right) \approx 2.410142\dots
\]
\end{theorem}

\begin{proof}
If $p$ is prime, we apply Prop. \ref{prop:patrones}: $\Omega(p)=0$ and $\Omega(p \cdot 2^j) = 2j$ for $j \ge 1$.
We develop the terms of the sum $T(p)$:
\begin{itemize}
    \item $k=0$: empty term = 1.
    \item $k=1$: $\frac{1}{1+\Omega(p)} = \frac{1}{1}$.
    \item $k=2$: $\frac{1}{1} \cdot \frac{1}{1+\Omega(2p)} = \frac{1}{1(1+2)} = \frac{1}{3}$.
    \item $k=3$: $\frac{1}{3} \cdot \frac{1}{1+\Omega(4p)} = \frac{1}{3(1+4)} = \frac{1}{15}$.
\end{itemize}
The denominator of the general term is the product of consecutive odd numbers (double factorial), which can be rewritten as $(2k-1)!! = \frac{(2k)!}{2^k k!}$.
Therefore, the series converges to:
\[
T(p) = \sum_{k=0}^{\infty} \frac{2^k k!}{(2k)!}.
\]
This series corresponds exactly to the Taylor expansion of the normalized $\operatorname{erf}(z)$ function evaluated at $z=1/\sqrt{2}$.
\end{proof}

\begin{proposition}[Maximum Entropy at $n=4$]
For the base case $n=4$, the function recovers Euler's number:
\[
T(4) = e.
\]
\end{proposition}

\begin{proof}
For $n=4$, we know from Prop. \ref{prop:patrones} that $\Omega(4 \cdot 2^j) = \Omega(2^{j+2}) = j$.
Substituting into the series, the product of the denominators generates the factorial sequence:
\[
\prod_{j=0}^{k-1} (1+j) = k!
\]
Therefore:
\[
T(4) = \sum_{k=0}^{\infty} \frac{1}{k!} = e.
\]
\end{proof}

\begin{theorem}[Damping Limit in Perfect Numbers]
Let $N_p = 2^{p-1}(2^p-1)$ be a sequence of even perfect numbers generated by Mersenne primes. Then, when $p \to \infty$, the iterated resonance collapses to the absolute minimum:
\[
\lim_{p \to \infty} T(N_p) = 1.
\]
\end{theorem}

\begin{proof}
We know that for an even perfect number $N_p$, $\Omega(N_p) = 2(p-1)$ (demonstrated in Section 7, but derived from identities in Prop. \ref{prop:patrones}).
The function $T(N_p)$ starts with the unit term, followed by terms damped by $\Omega$:
\[
T(N_p) = 1 + \frac{1}{1+\Omega(N_p)} + \frac{1}{(1+\Omega(N_p))(1+\Omega(2N_p))} + \dots
\]
The first fractional term is $\frac{1}{1 + 2(p-1)} = \frac{1}{2p-1}$.
Since $\Omega(n)$ is positive and increasing under doubling, all subsequent terms are smaller than this first term.
When $p$ grows (Mersenne primes are large), the denominator $2p-1$ tends to infinity, making $\frac{1}{2p-1} \to 0$. Consequently, the entire tail of the series vanishes, and $T(N_p) \to 1$.
\end{proof}

\begin{remark}
This result indicates that perfect numbers act as resonance "sinks". Their structure is so stable that any attempt to excite them via doubling (the process $T(n)$) is immediately damped by an immense initial divisor density.
\end{remark}

\subsection{Synthesis of the Resonance Spectrum}

The function $T(n)$ allows us to classify integers according to their "background energy" or capacity to sustain a resonant structure under iteration.

\begin{center}
\begin{tabular}{lccc}
\toprule
\textbf{Number Type} & \textbf{Symbol} & \textbf{Series Expression} & \textbf{Characteristic Value} \\ 
\midrule
Base Composite & $T(4)$ & $\displaystyle \sum_{k=0}^{\infty}\frac{1}{k!}$ & $e \approx 2.71828$ \\[10pt]
Prime Number & $T(p)$ & $\displaystyle \sum_{k=0}^{\infty}\frac{2^{k}\,k!}{(2k)!}$ & $\mathcal{T}_p \approx 2.41014$ \\[10pt]
Even Perfect ($p \to \infty$) & $T(N_{perf})$ & $\displaystyle 1 + \mathcal{O}(p^{-1})$ & $\to 1$ \\
\bottomrule
\end{tabular}
\end{center}

\begin{remark}
The spectrum of $T(n)$ is bounded. The value $e$ represents maximum natural growth (maximum structural entropy), while the value $1$ represents total stability (crystallization). Prime numbers occupy a stable intermediate band, governed by Gaussian statistics.
\end{remark}

% ------------------------------------------------------------
% SECTION 6 
% ------------------------------------------------------------
\section{The Frequency Seed $\Lambda_{MF}$ and the Zeta Identity}

We finalize the analytical part by decomposing $\Omega(n)$ into its atomic components. Instead of analyzing the multiplicative structure case by case, we use the algebra of Dirichlet series to isolate pure resonance information.

\begin{definition}[Frequency Seed]
Let $\Lambda_{MF}$ be the Dirichlet inverse of $\Omega$ normalized. We define their relationship via convolution:
\[
\Lambda_{MF} = \Omega * \mu \quad \iff \quad \Omega(n) = \sum_{d|n} \Lambda_{MF}(d).
\]
In the complex frequency domain, this implies that the generating series of the seed is the quotient of the series:
\[
L(s, \Lambda_{MF}) = \frac{\mathcal{D}_{\Omega}(s)}{\zeta(s)},
\]
where $\mathcal{D}_{\Omega}(s) = \sum \Omega(n)n^{-s}$.
\end{definition}

The following theorem establishes the closed form of this series, which will allow us to deduce the seed values directly.

\begin{theorem}[Zeta Spectral Identity]
The generating series of the seed $\Lambda_{MF}$ satisfies the following exact identity in relation to the Riemann Zeta function:
\[
\boxed{ L(s, \Lambda_{MF}) = (2 - 2^{-s})\zeta(s) - 4 }
\]
\end{theorem}

\begin{proof}
We start from the equivalence demonstrated in Section 2: $\Omega(n) = d(2n) - 4$.
We construct the Dirichlet series for $\Omega(n)$ by analyzing its two components separately.
\begin{enumerate}
    \item \textbf{Component $d(2n)$:} We use the arithmetic identity for the shifted divisor:
    \[
    d(2n) = 2d(n) - d(n/2)
    \]
    where we assume $d(x)=0$ if $x \notin \mathbb{Z}$. In the language of Dirichlet series, the displacement $n/2$ is equivalent to a multiplication by $2^{-s}$. Since the generating series of $d(n)$ is $\zeta^2(s)$, we have:
    \[
    \sum_{n=1}^{\infty} \frac{d(2n)}{n^s} = 2\zeta^2(s) - 2^{-s}\zeta^2(s) = (2 - 2^{-s})\zeta^2(s).
    \]
    
    \item \textbf{Constant Component $-4$:} The constant function $f(n)=-4$ generates the series $-4\zeta(s)$.
\end{enumerate}

Combining both terms, the total series for $\Omega(n)$ is:
\[
\mathcal{D}_{\Omega}(s) = (2 - 2^{-s})\zeta^2(s) - 4\zeta(s).
\]
Finally, to obtain the series of the seed $L(s, \Lambda_{MF})$, we apply Möbius inversion, which in this domain corresponds simply to dividing by $\zeta(s)$:
\[
L(s, \Lambda_{MF}) = \frac{(2 - 2^{-s})\zeta^2(s) - 4\zeta(s)}{\zeta(s)} = (2 - 2^{-s})\zeta(s) - 4.
\]
\end{proof}

Once the analytical identity is established, the discrete values of the seed emerge as the coefficients of the series expansion.

\newpage

\subsection{Integral Representation and Exact Balance Condition}

To analyze the asymptotic behavior, we apply the Abel summation formula. Due to the initial discontinuity at $n=1$, we isolate this term to ensure uniform convergence of the residue.

\begin{definition}[Strict Oscillatory Residue]
We define the error function $R(x)$ for the accumulated sum of the seed starting from $n=2$. We model the accumulated sum as linear growth plus an oscillatory term:
\[
\sum_{2 \le n \le x} \Lambda_{MF}(n) = 1.5(x-1) + R(x)
\]
Where $R(x)$ is a perfectly bounded step function, defined by parity:
\[
R(x) = \frac{1}{2}(-1)^{\lfloor x \rfloor - 1} \implies |R(x)| \le 0.5 \quad \forall x \ge 2.
\]
\end{definition}

This definition corrects deviations at the origin and allows for an exact formulation via integration by parts.

\begin{theorem}[Rectified Integral Representation]
For all $s$ with $\operatorname{Re}(s) > 1$, the generating function of the seed satisfies the following exact identity:
\[
L(s, \Lambda_{MF}) = -2 + \underbrace{1.5 \cdot 2^{-s} \left( \frac{s+1}{s-1} \right)}_{\text{Structural Component}} + \underbrace{s \int_{2}^{\infty} \frac{R(x)}{x^{s+1}} \, dx}_{\text{Oscillatory Component}}
\]
\end{theorem}

\begin{proof}
We decompose the Dirichlet sum separating the term $n=1$:
\[
L(s) = -2 + \sum_{n=2}^{\infty} \frac{\Lambda_{MF}(n)}{n^s}
\]
We apply the Abel summation formula. The integral of the linear part $1.5(x-1)$ generates, when evaluated rigorously between 2 and $\infty$, the structural term with the pole at $s=1$:
\[
s \int_{2}^{\infty} \frac{1.5(x-1)}{x^{s+1}} \, dx = 1.5 \cdot 2^{-s} \left( \frac{s+1}{s-1} \right)
\]
The remaining term is the integral of the residue $R(x)$, resulting in the theorem's identity.
\end{proof}

\begin{lemma}[Zero Balance Condition]
Starting from the spectral identity $(2-2^{-s})\zeta(s) - 4 = L(s, \Lambda_{MF})$, we know that $\zeta(s)=0$ if and only if $L(s, \Lambda_{MF}) = -4$. This implies that the total balance of the system must nullify:
\[
L(s, \Lambda_{MF}) + 4 = 0
\]
Substituting the integral representation:
\[
\left( -2 + 1.5 \cdot 2^{-s} \left( \frac{s+1}{s-1} \right) + s \int_{2}^{\infty} \frac{R(x)}{x^{s+1}} \, dx \right) + 4 = 0
\]
Simplifying, we obtain the fundamental equilibrium equation:
\[
2 + 1.5 \cdot 2^{-s} \left( \frac{s+1}{s-1} \right) + s \int_{2}^{\infty} \frac{R(x)}{x^{s+1}} \, dx = 0
\]
\end{lemma}

\begin{theorem}[Dynamic Stability Criterion]
Let $\mathcal{I}_{osc}(s) = s \int_{2}^{\infty} \frac{R(x)}{x^{s+1}} \, dx$. The Riemann Hypothesis is true if, for any $s$ outside the critical line ($\sigma \neq 1/2$), the magnitude of the oscillation is incapable of satisfying the zero balance. That is, if the strict inequality holds:
\[
|\mathcal{I}_{osc}(s)| < \left| 2 + 1.5 \cdot 2^{-s} \left( \frac{s+1}{s-1} \right) \right|
\]
\end{theorem}

\begin{remark}
This formulation allows us to attack the Riemann Hypothesis from another perspective. By isolating the oscillation in a definite integral, we transform the search for zeros into a study of capacities: does the integral have enough "force" to break the equilibrium outside the critical line? The exact numerical coincidence in the known zeros tells us we are on the right path; this is not an approximate model, it is the real machinery of the Zeta function exposed piece by piece.
\end{remark}
\newpage

% ============================================================
% === PART II: HEURISTIC MODELS AND CONJECTURES ==============
% ============================================================
\part{Heuristic Models and Conjectures}
\begin{center}
    \small\textit{In this second part, we abandon strict deduction to delve into dynamic modeling. Using the analytical tools from Part I ($\Omega, T, \Lambda_{MF}$), we propose new heuristics to address open problems, interpreting arithmetic not as a static structure, but as a living system of accumulated tensions and punctual relaxations.}
\end{center}

% ------------------------------------------------------------
% SECTION 7
% ------------------------------------------------------------
\section{Resonance and Perfect Numbers}

Within the MFN framework, perfect numbers are not merely curiosities of the sum of divisors, but states of **total harmonic equilibrium**. They represent configurations where the internal multiplicative structure resonates in perfect phase with the magnitude of the number, nullifying the need for external "adjustments".

\begin{theorem}[Resonant Signature of Even Perfect Numbers]
Let $N$ be an even perfect number of the Euclidean form $N = 2^{p-1}(2^p-1)$, where $M_p = 2^p-1$ is a Mersenne prime. Then:
\[
\Omega(N) = 2(p-1).
\]
\end{theorem}

\begin{proof}
We identify $N$ with the general form $i \cdot 2^k$, where the odd part is $i = M_p$ (prime) and the exponent is $k = p-1$.
Applying Identity (d) from Proposition \ref{prop:patrones}:
\[
\Omega(N) = (k+2)d(i) - 4.
\]
Since $i$ is prime, $d(i)=2$. Substituting the values:
\[
\Omega(N) = (p-1+2)(2) - 4 = (p+1)(2) - 4 = 2p + 2 - 4 = 2(p-1).
\]
\end{proof}

\subsection{The Perfect Damping Constant ($C_{Perf}$)}

In Section 5, we demonstrated that for even perfect numbers, the iterated resonance function collapses asymptotically: $T(N) \to 1$. This implies that these numbers act as entropy sinks. However, for finite $p$, a residue exists. This suggests that the numerical universe assigns a finite "energy cost" to the existence of perfection.

\begin{definition}
We define the \textbf{Perfect Damping Constant} as the accumulated sum of the residual resonances of all even perfect numbers $N_k$:
\[
C_{Perf} = \sum_{k=1}^{\infty} (T(N_k) - 1).
\]
\end{definition}

\begin{remark}[The Energy Budget]
Numerical evaluations suggest that this series converges rapidly ($C_{Perf} \approx 0.863\dots$). Interpreting this physically is revealing: there is a limited "budget" of residual resonance available to form perfect structures. Perfection is not free; it consumes a defined portion of the arithmetic phase space.
\end{remark}

\subsection{Conjecture on Odd Perfect Numbers}

The existence of Odd Perfect Numbers (OPN) is one of the oldest unknowns. The MFN offers an argument based on **resonance economy**.

\begin{conjecture}[Non-existence via Resonant Cost]
There exists no odd integer $N$ such that $\sigma(N)=2N$.
\end{conjecture}

\begin{remark}[Justification: Symmetry and Dirtiness]
For an even perfect number, the resonance $\Omega(N)$ is generated by a "clean" and efficient structure (a Mersenne prime and a pure power of 2). In contrast, an OPN requires, according to Euler's form $N = p^k m^2$, a dense and "dirty" multiplicative structure.
We postulate that the combined resonance $\Omega(N_{odd})$ necessary to simulate perfection would exceed the allowed energy budget ($C_{Perf}$), or generate an instability in $T(N)$ that would prevent the necessary damping ($T(N) \not\to 1$). The geometry of odd numbers simply does not allow such a degree of symmetry without breaking resonant coherence.
\end{remark}

% ------------------------------------------------------------
% SECTION 8
% ------------------------------------------------------------
\section{Arithmetic Tension and the ABC Conjecture}

The ABC Conjecture explores the deep friction between multiplication (which creates structure) and addition (which destroys it). The MFN framework translates this problem into a dynamic of **tension and dissipation**.

\begin{definition}[Total Harmonic Tension]
For a coprime triple $(a, b, c)$ with $a+b=c$, we define the system tension as the sum of its individual resonances:
\[
\Omega_{ABC} = \Omega(a) + \Omega(b) + \Omega(c).
\]
\end{definition}

Recall that $\Omega(p^k) = 2(k-1)$. A number with high powers (small radical) possesses a very high $\Omega$. We define this as a state of "high tension" or low configurational entropy.

\begin{conjecture}[Additive Dissipation Principle]
The addition operation acts as a "mixing" operator that dissipates multiplicative structure. If $a$ and $b$ are high-tension states (high powers), their sum $c = a+b$ will collapse, with asymptotic probability 1, to a low-tension state (free of large powers).
\end{conjecture}

\begin{remark}[Geometric Impossibility]
The standard ABC Conjecture states that $\operatorname{rad}(abc)$ cannot be much smaller than $c$. In our geometric language: it is impossible to tessellate three polygons $(P_a, P_b, P_c)$ with ultra-high frequency subdivisions if their sides are additively linked. Addition destroys the "phase coherence" necessary to maintain high resonances simultaneously in all three terms. The arithmetic universe does not tolerate infinite concentrations of tension in an additive triple.
\end{remark}

% ------------------------------------------------------------
% SECTION 9
% ------------------------------------------------------------
\section{Spectral Dynamics and the Riemann Hypothesis}

We reach the heart of the model. If we visualize the sequence of integers not as a static list, but as an evolving temporal process, we can model the distribution of primes as a dynamic response to the accumulation of tension.

\subsection{The Arithmetic Seismograph $\Psi_E(n)$}

We define a theoretical instrument, the "Seismograph", which simulates the eternal competition between the generation of divisors (which charges the system with energy) and the appearance of primes (which discharge and stabilize it).

\begin{definition}[Dynamics of the Seismograph $\Psi_E$]
Let $\Psi_E(2) = 0$. For $n > 2$:
\[
\Psi_E(n) = 
\begin{cases} 
\Psi_E(n-1) + T(n) & \text{if } n \text{ is composite (Charge)}, \\[8pt]
\displaystyle \frac{\Psi_E(n-1)}{\mathcal{T}_p} & \text{if } n \text{ is prime (Discharge)}.
\end{cases}
\]
Where $\mathcal{T}_p \approx 2.4101\dots$ is the Gaussian harmonic constant derived in Section 5.
\end{definition}

\subsection{The Equilibrium Constant $\mathcal{K}_{MF}$}

The "characteristic impedance" of the system is not arbitrary. In Section 6, we found the seed identity $L(s, \Lambda_{MF}) = (2 - 2^{-s})\zeta(s) - 4$.
The mean dynamics of the system are dictated by the solution of the spectral equilibrium equation:
\[
(2 - 2^{-\mathcal{K}_{MF}})\zeta(\mathcal{K}_{MF}) = 4 \implies \mathcal{K}_{MF} \approx 1.5645\dots
\]

\subsection{Numerical Evidence: Regression to the Mean and Dynamic Confinement}

Computational analysis of the seismograph, extended up to $N=10^6$, confirms that the system dynamics are not divergent, but strongly confined by a logarithmic attractor. By fixing the impedance at its theoretical value derived from the seed identity ($\mathcal{K}_{MF} \approx 1.5645$), the behavior is described by a law of regression to the mean:

\[
\Psi_E(n) = \underbrace{\mathcal{K}_{MF} \ln(n)}_{\text{Theoretical Attractor}} + \underbrace{\epsilon_{dyn}(n)}_{\text{Dynamic Correction}}
\]

Where $\epsilon_{dyn}(n)$ is not a simple random error nor a rigid constant, but a term of \textbf{dynamic correction} reflecting the elasticity of the system:

\begin{enumerate}
    \item \textbf{The Logarithmic Attractor ($\mathcal{K}_{MF}$):}
    The curve $\mathcal{K}_{MF} \ln(n)$ acts as the center of gravity of the system. Numerical validation shows that the asymptotic slope of the seismograph converges exactly to this value. Any deviation from this curve generates a tension that the system seeks to correct asymptotically.
    
    \item \textbf{System Inertia and Excitation ($\epsilon_{dyn}$):}
    We observe that the term $\epsilon_{dyn}(n)$ fluctuates with a local positive bias. This does not imply an added structural constant, but is a consequence of \emph{dynamic inertia}: since charge events (composite numbers) are much more frequent than discharge events (prime numbers, density $\sim 1/\ln n$), the system spends most of the time in "excitation" states above the attractor. However, the restoring force of primes guarantees that the system always orbits in the vicinity of the theoretical curve, never escaping.
\end{enumerate}

\begin{conjecture}[Error Confinement]
We postulate that the dynamic correction term $\epsilon_{dyn}(n)$ is strictly confined by the scale of the system. That is, the arithmetic tension never breaks the barrier imposed by the seed:
\[
\epsilon_{dyn}(n) = \Psi_E(n) - \mathcal{K}_{MF}\ln(n) = O(n^{1/2+\delta})
\]
This implies that fluctuations in the distribution of primes are the mechanism of \emph{regression to the mean} necessary to maintain the spectral equilibrium determined by $(2-2^{-s})\zeta(s)=4$.
\end{conjecture}

\begin{remark}[Arithmetic Elasticity]
Under this interpretation, the system of integers behaves like an elastic medium. The accumulation of divisors stretches the numerical "spring" (moving it away from the mean), and prime numbers are the breaking or relaxation points that allow the system to return towards its base state of minimum energy ($\mathcal{K}_{MF}\ln n$).
\end{remark}

% ============================================================
% === FINAL DISCUSSION =======================================
% ============================================================
\newpage
\section{Discussion and Paths Forward}

The Frequency Model of Numbers (MFN) has traced a conceptual arc from elementary geometry ($m=2n/k$) to the frontiers of the Zeta function. What emerges is not just a collection of formulas, but a coherent vision: integers possess an internal vibrational structure governed by deterministic laws of parity.

We have seen how the exact identity $\Omega(n) = d(2n)-4$ acts as a bridge between worlds, and how the Frequency Seed $\Lambda_{MF}$ reduces the apparent chaos of divisors to a simple atomic sequence ($\{-2, 1, 2\}$), whose generating series $(2-2^{-s})\zeta(s)-4$ dictates the fundamental impedance $\mathcal{K}_{MF}$ of the system.

To consolidate this theory and move from heuristic to rigorous proof, we propose the following roadmap:

\begin{enumerate}[label=\textbf{\arabic*.}]
    \item \textbf{Analysis of Dynamic Inertia:} Instead of looking for an arbitrary correction constant, the statistical nature of $\epsilon_{dyn}(n)$ must be investigated. It is necessary to formally model the probability of return to the mean, demonstrating that the frequency of primes is the minimum necessary to prevent the inertia of composites from causing the system to diverge.
    
    \item \textbf{Formalization via Tauberian Theorems:} It is required to apply complex analysis to prove that the charge/discharge dynamics defined by the attractor $\mathcal{K}_{MF}$ effectively converge to the distribution of primes. The link of $\Lambda_{MF}$ with $\zeta(s)$ facilitates the use of Perron's formula to rigorously bound partial sums.
    
    \item \textbf{Study of the Seed:} The function $\Lambda_{MF}$ is the most solid finding of the work. Investigating the autocorrelation properties of this deterministic sequence could offer a new, purely arithmetic path to bound the error term in the Prime Number Theorem, interpreting the Riemann Hypothesis as a stability problem of coupled oscillators.
\end{enumerate}

In conclusion, the geometric-harmonic approach offers a refreshing perspective: prime numbers are not random anomalies, but the \textbf{necessary dissipators} that, through a continuous process of regression to the mean, maintain the resonant stability of the arithmetic universe.

\end{document}